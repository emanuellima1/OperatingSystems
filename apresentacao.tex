\documentclass{beamer}

\usepackage[utf8]{inputenc}
\usepackage[portuguese]{babel}

\title[EP1]{Shell e simulador de processos}
\author{Emanuel Lima e João Seckler}
\date{outubro de 2020}

% Cores mais padrão (preto e branco, basicamente)
% \usecolortheme{dove}

\begin{document}
\frame{\titlepage}

\section{Implementação}
\begin{frame}[fragile]
  \frametitle{O Shell bccsh}

  É basicamente composto pelo seguinte laço:

  \begin{verbatim}
  enquanto comando := readline(prompt) não é EOF
    se comando é um embutido (mkdir, kill, ln ou cd)
      executa o embutido
    senão
      divide o comando em palavras
      se (fork())
        wait()
      senão
        execv(palavras)

  \end{verbatim}

  Onde \verb/readline/ é definida pela biblioteca \verb/readline/,
  prompt é construída como especificado no EP. Essa estrutura omite
  alguns detalhes como manipulação de sinais e construção do prompt.

\end{frame}

\begin{frame}[fragile]
  \frametitle{O simulador de escalonamento de processos}
  Variáveis globais:
  \begin{itemize}
    \item \verb\mutexv\: vetor de \verb/pthread_mutex_t/, um para cada processo
    \item \verb\indices\: vetor de \verb/int/, um para cada processo,
      que identifica qual posição da lista \verb\mutexv\ acessar
  \end{itemize}
  Comunicação entre o simulador e os processos:
  \begin{itemize}
    \item Criação do processo: \verb\pthread_create\
    \item Interrupção do processo: \verb\pthread_mutex_lock\
    \item finalização do processo: \verb\pthread_cancel\ (a thread roda
      \verb\pthread_testcancel\ a cada laço)
  \end{itemize}
  Esse protocolo de comunicação permite às threads usarem o mesmo código
  nas três implementações de escalonador. O controle de tempo é todo
  feito no escalonador.
\end{frame}

\begin{frame}[fragile]
  \frametitle{O simulador de escalonamento de processos}
  \textbf{O que há de comum} nas três implementações: um laço que roda
  até a lista de processos acabar (e até a fila ficar vazia, quando
  houver uma). A cada iteração do laço, o programa vê qual é o próximo
  evento, entre:

  \begin{itemize}
    \item a chegada de um novo processo (em cujo caso, passa-se para o
      próximo item da lista de processos)
    \item o fim de um novo processo
    \item preempção (no caso do \verb/STRN/, pela chegada
      de um novo processo, no caso do \verb/round robin/, pelo fim de um
      quantum)
  \end{itemize}

  Em cada caso, o programa espera o tempo até o evento ocorrer e
  cuida dele.

  \begin{itemize}
    \item \verb/STRN/: fila de prioridade
    \item \verb/round robin/: fila simples.
  \end{itemize}

\end{frame}

\section{Resultados de experimentos}

\end{document}
